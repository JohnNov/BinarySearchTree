\title{\vspace*{2in}\Huge Coaxing the Leviathan}

\documentclass[12pt]{article}
\usepackage{authblk}
\usepackage{multicol}
\usepackage{sectsty}
\usepackage[parfill]{parskip}
\usepackage[margin=0.8in]{geometry}
\usepackage{comment}
%\usepackage{fontspec}

%Inverted colors begin
\usepackage{xcolor}

\pagecolor[rgb]{0.01,0.01,0.01}
\color[rgb]{1,1,1}
%Inverted colors end 

\subsectionfont{\normalfont\large\underline}

\renewcommand\Authfont{\fontsize{18}{8.4}\selectfont}



\begin{document}
\author{}
\date{}
\maketitle
\pagenumbering{gobble}
\begin{quote}{\large}
"There is clearly an intelligence at work that is far greater than the mind."\\
\textemdash Eckhart Tolle
\end{quote}

\pagebreak

\section*{About}

After analyzing every great and weak moment of my life (via the Bad-Ass Delta list), I realized that the state of flow was either acting perfectly or missing in action respectively.  It seems as if the flow state is merely the uninterrupted subconscious - The Leviathan.  We will let the data speak for itself: The flow state is the key to a strong personality.  We will take action and shift behavior to gain the ability to coax the Leviathan at a moments notice.  
\\
\\

\begin{multicols}{2}
\subsection*{The Good:}

\begin{enumerate}
	\item Authenticity
	\item The will (not desire!) to hurt
	\item Effortlessness
	\item Timelessness
\end{enumerate}


\subsection*{The Bad:}

\begin{enumerate}
	\item Fear (Self-Awareness)
	\item Inner Critic (Self-Awareness)

\end{enumerate}
\vfill\null
\end{multicols}

\begin{comment}


\section*{Theory}
It seems like there is a strong correlation with these traits and the flow state.  We want to maximize the good traits and minimize the bad traits.  

\subsection*{Effortlessness:}
To be effortless we must remove effort.  We remove effort by removing the fear of losing and desire to win.  The ego must be let go.  To lessen the ego we must stop feeding it.  

\subsection*{Authenticity:}
This is a byproduct of flow state.  How could one be unauthentic if they are acting effortlessly in the moment?  To be authentic is not to be unauthentic.  


\subsection*{The will (not desire!) to hurt:}
To hurt is not the goal.  It isn't the key to our emotional salvation.  When we choose to hurt it will be done out of fun righteousness.  

Some conflict in life is necessary.

\subsection*{Timelessness:}
\bigskip

\subsection*{Fear:}

\bigskip

\subsection*{Inner Critic:}

\end{comment}


\pagebreak
\begin{samepage}
\section*{Philosophies}
We want to adhere to philosophies that would addresses these traits the most and then master them.  
\newline 

{\large I. The process is the goal.\par}

\textbf{Addresses}
\begin{list}{$\circ$}{}
	\item Fear (We won't care if we fail)
	\item Effortlessness (Removing the emotional pressure of achieving a desired goal can only help).  
\end{list}

\noindent\makebox[\linewidth]{\rule{\paperwidth}{0.4pt}}
\smallskip


{\large II. Accept yourself.\par}

\textbf{Addresses}
\begin{list}{$\circ$}{}
	\item Inner Critic
\end{list}

\noindent\makebox[\linewidth]{\rule{\paperwidth}{0.4pt}}
\smallskip

{\large III. Be in the Now.\par}

\textbf{Addresses}
\begin{list}{$\circ$}{}
	\item Timelessness
\end{list}

\noindent\makebox[\linewidth]{\rule{\paperwidth}{0.4pt}}

\smallskip
{\large IV. Remember that the fear is the worst part.\par}
\textbf{Addresses}
\begin{list}{$\circ$}{}
	\item Fear 
\end{list}


\noindent\makebox[\linewidth]{\rule{\paperwidth}{0.4pt}}

\smallskip
{\large V. Have fun with your Righteousness!}
\begin{list}{$\circ$}{}
	\item Authenticity
	\item The will (not desire!) to hurt
	\item Effortlessness 
\end{list}





\end{samepage}



\pagebreak


\section*{Meditations}

Analysis of the Philosophies.  

\subsection*{The process is the goal:}


\pagebreak

\subsection*{Accept Yourself}

There are many different types of strong and respectable personalities

\textit{Accepting it feels hurtful, as if a voice coming from a third party is confirming my fears.  "That's right you little bitch."}

\textit{This is your core belief talking.  "We're not *blank* enough."}

\textit{Out of all of the harm traits, Dr. Alpern said that the Inner Critic was the most harmful and the most susceptible to change.}

\textit{"The biggest issue most people have is a lack of self-acceptance."}

\textit{These are mitigated by not caring about being *blank* enough.}

\textit{Don't be textbook alpha.  Be John Alpha.}

\pagebreak




\subsection*{Be in the Now}


\textit{Stop reminiscing, good or bad, past or future.  You do not need to do that.  Find joy in the Now.  This is the cost of being great.}

\textit{You aren't allowed this anymore.  You can't do it.}

\textit{You must consciously wage war against the past and force yourself to react to the present moment. Be ruthless on yourself; do not repeat the same tired methods.}

\textit{There is never any value in fighting the last war.}

\textit{You have a certain amount of energy and headspace.  Why waste it on reliving the same moments?}
\pagebreak



\subsection*{Remember: The fear is the worst part.}

\textit{Do not be fooled by other people's aura.  Ask what and why they are really doing it.  Then handle it however you want.}

\pagebreak

\subsection*{Have fun with your Righteousness!.}
\textit{Don't be confident to achieve a particular tactical goal or struggle.  Be confident for the \textbf{fun} of it.  The struggle will resolve itself.}

\begin{comment}



Look at things as they are, not as your emotions paint them.  The only remedy is to be aware that the pull of emotion is inevitable , to notice it when it is happening , and to compensate for it . When you have success , be extra wary . When you are angry , take no action . When you are fearful , know you are going to exaggerate the dangers you face . 

Remember: Every action and point of view can be attacked.  There's no such thing as a no critic frame.  


3. 
You should just stay in the Now.  It's way more fun.  Or you'll have more fun.  
For the Now page in the leviathon you should argue that the less you stay in the now the less you'll achieve clutch 


4. Don't be confident to achieve a particular tactical goal or struggle.  Be confident for the \textbf{fun} of it.  The struggle will resolve itself.


 5. Do not be fooled by other people's aura.  Ask what and why they are really doing it.  Then handle it however you want.

		* The most aggressive guys are also the most sensitive and afraid.  




It is a disease to be obsessed by the thought of winning. It is also a disease to be obsessed by the thought of employing your swordsmanship. So it is to be obsessed by the thought of using everything you have learned, and to be obsessed by the thought of attacking. It is also a disease to be obsessed and stuck with the thought of ridding yourself of any of these diseases. A disease here is an obsessed mind that dwells on one thing. Because all these diseases are in your mind, you must get rid of them to put your mind in order.

Greene, Robert. The 33 Strategies of War (Joost Elffers Books) . Penguin Publishing Group. Kindle Edition. 




War is a necessary part of life.  It must be done.  
Repressing your anger , avoiding the person threatening you , always looking to conciliate — these common strategies spell ruin . Avoidance of conflict becomes a habit , and you lose the taste for battle .

Once you suspect you are dealing with a Napoleon , do not lay down your arms or entrust them to someone else . You are the last line of your own defense .

Save your carefulness for the hours of preparation , but once the fighting begins , empty your mind of doubts .



6. 


My fear is vastly disproportionate to the actual danger.  Once you think back and remember everything objectively you'll realize this is the truth.  


\end{comment}


\end{document}
